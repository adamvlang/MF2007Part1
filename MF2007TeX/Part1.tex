% vim:tw=72 sw=2 ft=tex
%         File: Part1.tex
% Date Created: 2016 Jan 22
%  Last Change: 2016 Jan 22
%     Compiler: pdflatex
%       Author: Lamn
\documentclass[12pt,a4paper]{article}
\usepackage{amsmath, amssymb}
\usepackage[utf8]{inputenc}
\usepackage[T1]{fontenc}
\usepackage[english]{babel}
\usepackage{graphicx}
\usepackage{mathrsfs}

\title{MF2007 Workshop Part1}
\author{Adam Lang, Andreas Fr\"{o}derberg, Gabriel Andersson Santiago}

\newcounter{eq}
\stepcounter{eq}

\newcommand{\eq}[1]{
\begin{equation}
        #1
\end{equation}
    }

\begin{document}
\maketitle
\section{Ex. 1}
  The model can be described with the following differential equation,
  \eq{
  f-d\dot{x}=m\ddot{x}\Leftrightarrow \ddot{x}=\frac{1}{m}(f-d\dot{x})
  }
  In this system we have one energy storing element, the mass, $m$. The
  transfer function can be derived by first Laplace transforming the
  differential equation,
  \eq{
  \mathscr{L}\{\ddot{x}=\frac{1}{m}(f-d\dot{x})\}\Rightarrow s^2Y=(\frac{1}{m}(U-dsY)
  }
  Then can the transfer function be found as,
  \eq{
  G(s)=\frac{1}{ms^2+ds}
  }
  We can also derive a state space model from the differential equations
  where,
  \eq{
  \begin{bmatrix}
    x_1 \\
    x_2 \\
  \end{bmatrix}
  =
  \begin{bmatrix}
    x \\
    \dot{x} \\
  \end{bmatrix},
  }
  so that, 
  \eq{
  \mathbf{\dot{x}}=
  \begin{bmatrix}
    x_2\\
    \frac{1}{m}(F-dx_2)\\
  \end{bmatrix}.
  }
  This will then give us the state space model as,
  \eq{
  \begin{cases}
  \mathbf{\dot{x}}=
  \begin{bmatrix}
    0 & 1 \\
    0 & -\frac{d}{m} \\
  \end{bmatrix}
  \mathbf{x}+
  \begin{bmatrix}
    0 \\
    \frac{1}{m} \\
  \end{bmatrix}
  \mathbf{u}\\
  \vspace{0.02cm} \\
  \mathbf{y}=
  \begin{bmatrix}
    0 & 1\\
  \end{bmatrix}
  \mathbf{x}
\end{cases}
  }



\end{document}
